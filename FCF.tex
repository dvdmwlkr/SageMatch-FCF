\documentclass[12pt]{article}

\usepackage{amsmath, amssymb}
\usepackage[utf8]{inputenc}
\usepackage[english]{babel}
\usepackage{csquotes}
 
\usepackage[a4paper, total={6in, 10in}]{geometry}
\usepackage{bibstuff} 
\addbibresource{peters.bib}

\newtheorem{theorem}{Theorem}
\newtheorem{acknowledgement}[theorem]{Acknowledgement}
\newtheorem{algorithm}[theorem]{Algorithm}
\newtheorem{axiom}[theorem]{Axiom}
\newtheorem{case}[theorem]{Case}
\newtheorem{claim}[theorem]{Claim}
\newtheorem{conclusion}[theorem]{Conclusion}
\newtheorem{condition}[theorem]{Condition}
\newtheorem{conjecture}[theorem]{Conjecture}
\newtheorem{corollary}[theorem]{Corollary}
\newtheorem{criterion}[theorem]{Criterion}
\newtheorem{definition}[theorem]{Definition}
\newtheorem{example}[theorem]{Example}
\newtheorem{exercise}[theorem]{Exercise}
\newtheorem{lemma}[theorem]{Lemma}
\newtheorem{notation}[theorem]{Notation}
\newtheorem{problem}[theorem]{Problem}
\newtheorem{proposition}[theorem]{Proposition}
\newtheorem{remark}[theorem]{Remark}
\newtheorem{solution}[theorem]{Solution}
\newtheorem{summary}[theorem]{Summary}
\newenvironment{proof}[1][Proof]{\textbf{#1.} }{\ \rule{0.5em}{0.5em}}
%\input{tcilatex}

\begin{document}

\title{Factorisation and Continued Fractions}
\author{Peter Walker}
\maketitle

\begin{abstract}
In an early paper \cite{vonSchafgotsch:1786:FVS} on factorisation of
integers, the author proposes an elementary method which we develop here by
means of continued fractions.. Its success rate (the proportion of integers
non-trivially factorised) is surprisingly high, at around \ 80\%, and we are
able to give an explicit description of the cases in which the method fails.

We then give a number of improvements and extensions of the method which
increase both its success rate and its speed of execution. In particular we
introduce the elementary quadratic domains $\mathbf{Z}(i),\mathbf{Z}(\sqrt{-2%
})$ and $\mathbf{Z}(2)$ as areas in which factorisation of rational integers
may be carried out successfully.

Subject Classification: 11A51, 11A55, 11R11
\end{abstract}

\section{Introduction and a simple algorithm\label{int}}

Fermat's method of factorisation of a given integer by expressing it as the
difference of two squares (or more generally as a divisor of the difference
of two squares) is one of the oldest in number theory. An early example of
its use was given by F. von Schafgotsch, \cite{vonSchafgotsch:1786:FVS} in
which for given odd integer $n$ we consider integers $a,b,x,$ which satisfy $%
an=b\left( x^{2}-1\right) $ and $n,a,b$ are relatively prime. Then the
greatest common divisors $n_{1}=\gcd \left( n,x+1\right) ,$ and $n_{2}=\gcd
\left( n,x-1\right) $ give a factorisation $n=n_{1}n_{2}.$ Unfortunately no
practicable method for finding suitable $a,b,x$ is given in \cite%
{vonSchafgotsch:1786:FVS}, but it is not difficult to give such a method by
using the theory of continued fractions.

We use the standard notation for the continued fraction expansion of the
square root of a given integer, as detailed for instance in \cite%
{Davenport:1970:THA}.

\textbf{Example 1}\label{e1} Let $n=91,$ with continued fraction expansion 
\begin{equation*}
\sqrt{91}=\left[ 9,1,1,5,1,5,1,1,18,....\right] .
\end{equation*}%
The sequence of convergents is%
\begin{equation*}
\left( 9/1,10/1,19/2,105/11,124/13,725/76,849/89,1574/165,...\right) 
\end{equation*}%
Write $x$ for the numerator of the last entry here, corresponding to the
term before the occurrence of $18=2\left[ \sqrt{91}\right] .$ Here $x=1574.$
Then the greatest common divisors $\gcd (n,x-1)=\gcd \left( 91,1573\right)
=13$ and $\gcd (n,x+1)=\gcd \left( 91,1575\right) =7$ give $91=13.7$ as the
product of relatively prime odd integers. $\blacklozenge $ \ (In addition
to the usual $\blacksquare $ to indicate the end of a proof, we shall use $%
\blacklozenge $ to indicate the end of the discussion following an example.)

To investigate the extent to which this method works more generally, we
shall need the following result from the classical theory of continued
fractions \cite[Chapter IV]{Davenport:1970:THA}.

\begin{theorem}
\label{t1}Let $n$ be a positive non-square odd integer. Write $n$ as a
continued fraction $\sqrt{n}=\left[ a_{0},a_{1},...,a_{k},...\right] $ where 
$a_{1},...,a_{k}$ is the periodic part, $a_{0}=\left\lfloor \sqrt{n}%
\right\rfloor $ and $a_{k}=2a_{0}.$ $k$ is called the length of the
continued fraction (strictly, the length of the periodic part). Let $x/y$
denote the last but one convergent here, corresponding to the term $a_{k-1}.$
Then 
\begin{equation}
\left( x+y\sqrt{n}\right) \left( x-y\sqrt{n}\right) =x^{2}-ny^{2}=\left(
-1\right) ^{k}.  \label{inta}
\end{equation}
\end{theorem}

When $k$ is even, the equation is 
\begin{equation}
x^{2}-ny^{2}=1  \label{intb}
\end{equation}%
and it is known that $(x,y)$ determines the smallest solution of this
equation and that all other solutions are given by $\left( x_{i}+y_{i}\sqrt{n%
}\right) =\left( x+\sqrt{n}\right) ^{i},i\in \mathbf{Z.}$Hence we have the
following corollary:

\begin{corollary}
\label{t2}Let $n$ be a positive non-square odd integer for which the
continued fraction for $\sqrt{n}$ has even length. Let $x,y$ be as in the
preceding theorem, and let $n_{1},n_{2}$ equal $\gcd \left( n,x-1\right) $
and $\gcd \left( n,x+1\right) $ respectively. Then $n=n_{1}n_{2}$ gives a
factorisation of $n$ which is non-trivial unless $n$ divides either $x-1$ or 
$x+1.$
\end{corollary}

\begin{proof}
Write (\ref{intb}) as $ny^{2}=\left( x-1\right) \left( x+1\right) $ in which 
$x\pm 1$ can have no common divisors other than possibly $2$. Hence each odd
prime divisor of $n$ must divide exactly one of $x\pm 1$ and the result
follows.
\end{proof}

We deduce that the method succeeds except in one or other of two cases.
Firstly if $k$ is odd, or equivalently if $a_{1},...,a_{k-1}$ has no central
term, then we arrive at $y^{2}n=x^{2}+1$ and we can not factorise in the way
required (these cases are dealt with in section \ref{gi} on Gaussian
integers). We shall call this a Type I failure. Secondly, when $k$ is even,
the resultant factorisation may be trivial in that one or other of $%
n_{1},\,n_{2}$ may equal $1.$ This happens if and only if one of $x\pm 1$ is
a multiple of $n;$ in particular this would happen if $n$ is prime. We shall
call this a Type II failure; integers of this type are dealt with in
sections \ref{zmt} and \ref{zpt}.

We illustrate these possibilities with some further examples.

\textbf{Example 2 }Let $n=85,$ $\sqrt{85}=\left[ 9;4,1,1,4,18,...\right] .$
Here $k=5$ is odd, and we have a Type I failure. $\blacklozenge $

\textbf{Example 3 }Let $n=187,$ $\sqrt{n}=\left[ 13;1,2,13,2,1,26,...\right]
.$ Here $k=6,$ and $x=1682.$ Then gcd$\left( n,x-1\right) =\gcd \left(
187,1681\right) =1$ and gcd$\left( n,x+1\right) =$gcd$\left( 187,1683\right)
=$ $187,$ giving $187=1\cdot 187,$ a Type II failure. $\blacklozenge $

\textbf{Example 4} For a slightly larger $n$, let $n=4725,$

$\,\sqrt{4725}=\left[ 68,1,2,1,4,1,2,1,136,...\right] ,$ \ $x=6049.\ $

Then gcd$\left( n,x-1\right) =\gcd \left( 4725,6048\right) =189$, gcd$\left(
n,x+1\right) =$gcd$\left( 4725,6050\right) =25,$ and hence $n=189.25.$ Then
the method can be repeated with $n=$ $189$ to give $189=3^{3}\cdot 7$ and we
end with $4725=3^{3}5^{2}7.$ $\blacklozenge $

Note that, as in this example, an odd prime power divisor of $n$ can divide
only one of $x-1$, $x+1$ so that, if successful, we are factoring into prime
powers rather than individual primes, see for instance \cite[1.7.3]%
{Cohen:1993:EF}.

We illustrate the method with some numerical results, found by David Walker
using \texttt{sagemath}, which illustrate the proportions of the various
outcomes and which are intended for comparison with other improvements to
come. The values of $n$ tested are those which are composite, not square and
not divisible by $2,3,5.$ (Thus the starting value is 77 which is the least
integer satisfying these conditions.) The times are of course highly
dedpendent on the machine and the program, and again are included only for
comparison with later methods.

\begin{center}
\begin{equation*}
\begin{tabular}{|l|l|l|l|l|}
\hline
Range of $n$ & Success & Type I fail & Type II fail & Time(s) \\ \hline
$77\leq n\leq 1,000$ & $73\left( 79.4\%\right) $ & $6\left( 6.5\%\right) $ & 
$13(14.1)$ & $0.04$ \\ \hline
$77\leq n\leq 2,000$ & $176(79.6\%)$ & $18\left( 8.1\%\right) $ & $27\left(
12.2\%\right) $ & $0.09$ \\ \hline
$77\leq n\leq 5,000$ & $502\left( 77.6\%\right) $ & $58\left( 9.0\%\right) $
& $87\left( 13.4\%\right) $ & $0.29$ \\ \hline
$77\leq n\leq 10,000$ & $1077\left( 76.4\%\right) $ & $134\left(
9.5\%\right) $ & $199\left( 14.1\%\right) $ & $0.66$ \\ \hline
$77\leq n\leq 20,000$ & $2301\left( 75.9\%\right) $ & $307\left(
10.1\%\right) $ & $423\left( 14.0\%\right) $ & $1.62$ \\ \hline
\end{tabular}%
\end{equation*}
\end{center}

When $k$ is odd we noted already that we have 
\begin{equation*}
y^{2}n=x^{2}+1
\end{equation*}%
and so $n$ is a divisor of a primitive (relatively prime) sum of squares. It
follows from the general theory of integers which are sums of squares %
\cite[Chapter V]{Davenport:1970:THA} that $n$ and all its \ prime divisors
are congruent to 1 mod $4$ and so are also sums of squares. This proves

\begin{proposition}
\label{p1}If the method fails because $k$ is odd (a Type I failure) then
every prime divisor of $n$ is congruent to $1$ mod $4.$ Equivalently if any
prime divisor of $n$ is congruent to $-1$ mod $4$ (and in particular if $n$
itself is congruent to $-1$ mod $4),$ then $k$ is even and the method will
give a factorisation (possibly trivial)\ of $n.$
\end{proposition}

This necessary conclusion is not sufficient: the first example to the
contrary is when $n=221=13.17$ \ and $k=6.$

\begin{proposition}
\label{p2}If $k$ is even and the factorisation is trivial (a Type II
failure), then $n\equiv -1$ mod $4.$ Equivalently if $k$ is even and $%
n\equiv 1$ mod $4$ then the method will succeed.
\end{proposition}

\begin{proof}
When $k$ is even we noted in $\left( \ref{intb}\right) $ that $%
y^{2}n=x^{2}-1.$ Hence if $y$ is odd then $x$ is even and so $x^{2}-1$ $%
\equiv -1$ mod $4.$ But as an odd square, $y^{2}\equiv 1$ mod $4$ and it
follows that $n\equiv y^{2}n=x^{2}-1$ $\equiv -1$ mod4.

The case in which $y$ is even while $x$ is odd presents more difficulty. We
begin with the observation that since $x-1,x+1$ are consecutive even numbers
then one has the form $2f$ for some odd integer $f,$ while the other has the
form $2^{j}g$ for some $j\geq 2$ and $g$ odd. Also we are assuming that $y$
is even, so $y=2^{i}s$ for $i\geq 1$ and $\ s$ is odd. Equating powers of $2$
in $\left( \ref{intb}\right) $ gives $2i=1+j$ so $j$ must be odd, $j\geq 3$
and $i\geq 2.$ Hence we have either

\begin{eqnarray*}
x-1 &=&2f\text{ and }\,x+1=2^{j}g=2^{2i-1}g,\text{ or} \\
x-1 &=&2^{2i-1}g\text{ and }x+1=2f.
\end{eqnarray*}

Hence either 
\begin{equation}
2^{2i-1}g-2f=2,\text{and so }f=4^{i-1}g-1  \tag{A}
\end{equation}

or 
\begin{equation}
2f-2^{2i-1}g=2,\text{ and so }f=4^{i-1}g+1.  \tag{B}
\end{equation}

Also if we substitute for $x-1,x+1,y$ in $\left( \ref{intb}\right) $ and
remove powers of $2$ we get 
\begin{equation*}
(2^{i}s)^{2}n=\left( 2f\right) \left( 2^{2i-1}g\right) ,\text{ and so }%
s^{2}n=fg
\end{equation*}%
where all $s,n,f,g$ are odd and $f,g$ are relatively prime. \ Since we know
that $n$ divides exactly one of $x-1$ \ or $x+1$ if follows that $n$ must
divide exactly one of $f,g$. \ Write $s_{1}=\gcd \left( s,f\right) $ and $%
s_{2}=\gcd \left( s,g\right) $ so when $n$ divides $f$ we have $s_{1}^{2}n=f$
and $s_{2}^{2}=g,$ or when $n$ divides $g$ we have $s_{1}^{2}=f$ and $%
s_{2}^{2}n=g.$ Hence exactly one of $f$,$g$ is a square.

This gives us four possibilities to be considered:

A1 when A above is satisfied and $g$ is square, A2 when A is satisfied and $%
f $ is square, B1 when B is satisfied and $g$ is square, and B2 when B is
satisfied and $f$ is square.

In case A1, condition A together with $g$ being square shows that $f$ is one
less than an even square and hence that $n\equiv f\equiv -1$ mod 4 as
required.

In case A2, condition A gives $f\equiv -1$ mod 4 while $f$ being square
gives $f\equiv 1$ mod 4, a contradiction.

In case B1, we have $f=4^{i-1}g+1$, $s_{1}^{2}n=f$ and $s_{2}^{2}=g.$ Hence%
\begin{eqnarray*}
s_{1}^{2}n=f=4^{i-1}g+1 &=&4^{i-1}s_{2}^{2}+1, \\
\left( 2^{i-1}s_{2}\right) ^{2}-s_{1}^{2}n &=&-1.
\end{eqnarray*}%
which is of the form $ny^{2}=x^{2}+1$. But as noted, following this equation
in the introduction, there are no such solutions when $k$ is even so case B1
is impossible.

Finally in case B2, we have $f=4^{i-1}g+1,\,s_{1}^{2}=f$ and $s_{2}^{2}n=g.$
Hence 
\begin{eqnarray*}
s_{1}^{2}=f=4^{i-1}g+1 &=&4^{i-1}s_{2}^{2}n+1 \\
s_{1}^{2}-\left( 2^{i-1}s_{2}\right) ^{2}n &=&1
\end{eqnarray*}%
and we have a solution $\left( s_{1},2^{i-1}s_{2}\right) $of equation $%
\left( \ref{intb}\right) .$ Here $s_{1}^{2}=f<fg=s^{2}n<y^{2}n$ since $%
y=2^{i}s$ for $i\geq 1.$ \ But form$,$ $y^{2}n=x^{2}-1$ we conclude that $%
s_{1}<x$ which contradicts the minimality of $\left( x,y\right) $ as a
solution of $\left( \ref{intb}\right) $ giving the required contradiction.
\end{proof}

Again the necessary conclusion is not sufficient since for instance $%
n=15,35,39$ have $n\equiv -1$ mod 4 and $k=2$ and all factorise successfully.

Combining the results of Propositions \ref{p1} and \ref{p2} gives

\begin{theorem}
\label{t3}If $n\equiv 1$ mod $4$ but has a divisor which is $\equiv -1$ mod $%
4$ then the method will give a non-trivial factorisation of $n.$ \ In
particular the method will succeed when $n$ is a product of distinct primes,
an odd number of which are $\equiv -1$ mod $4$.
\end{theorem}

\begin{proof}
From Proposition \ref{p1} $k$ must be even and so from Proposition \ref{p2}
the method will give a non-trivial factorisation of $n$
\end{proof}

The situation regarding the parity of $k$ is clearer when $n$ is prime.

\begin{corollary}
\label{t4}When $n$ is prime then
\end{corollary}

(i) $k$ is odd if and only if $n\equiv $ 1 mod 4, and

(ii) $k$ is even if and only if $n\equiv $ $-1$ mod 4.

\begin{proof}
To begin with, if $n$ is prime and $k$ is odd then from Proposition \ref{p1}%
, we find $n\equiv $ 1 mod 4 which is the direct half of (i). It also shows
that if $n\equiv $ $-1$ mod 4 then $k$ is even, which is the converse of
(ii).

n addition if $n$ is prime then the condition that $n|\left( x^{2}-1\right) $
should imply $n|\left( x-1\right) $ or $n|\left( x+1\right) $ is satisfied
automatically so the factorisation must be trivial and $n\equiv $ $-1$ mod 4
follows from Proposition \ref{p2} when $k$ is even. This is the direct half
of (ii) and also the converse of (i).
\end{proof}

\section{Looking for Squares\label{ls}}

To develop the method further, we shall need the customary notation 
\begin{equation*}
\left( A_{j}\right) ,\left( B_{j}\right) ,\left( C_{j}\right) ,\left(
P_{j}\right) ,\left( Q_{j}\right) 1\leq j\leq k,
\end{equation*}%
for the development of the continued fraction for the square root of a
positive non-square integer $n$ as found in for instance \cite[Ch. 10]%
{Bressoud:1989:IEF}. Here $\left( A_{j}\right) $ is the same as the sequence 
$\left( a_{j}\right) $ in Theorem \ref{t1} and $\left( P_{j}/Q_{j}\right) $
gives the sequence of convergents where $P_{k-1}/Q_{k-1}$ is denoted there
by $x/y$.

These sequences have the initial values 
\begin{eqnarray*}
A_{1} &=&n_{0}=\left\lfloor \sqrt{n}\right\rfloor , \\
B_{0} &=&0,\,B_{1}=n_{0}, \\
C_{0} &=&1,\,C_{1}=n-\left( n_{0}\right) ^{2}, \\
P_{0} &=&1,\,P_{1}=n_{0},
\end{eqnarray*}%
and for $j\geq 2$ satisfy the recurrences%
\begin{align}
\text{(i) }A_{j+1}& =\left\lfloor \frac{n_{0}+B_{j}}{C_{j}}\right\rfloor 
\text{ },  \label{ls1} \\
\text{(ii) }P_{j+1}& =P_{j-1}+A_{j+1}P_{j},  \notag \\
\text{(iii) }B_{j+1}& =A_{j+1}C_{j}-B_{j},  \notag \\
\text{(iv) }C_{j+1}& =C_{j-1}+A_{j+1}\left( B_{j}-B_{j+1}\right) .  \notag
\end{align}

An equivalent form for (iv) is $B_{j+1}^{2}+C_{j}C_{j+1}=n,$ from which (iv)
is deduced by putting $j$ for $j+1$ and subtracting.

The principal property of these sequences which we shall need, is that at
every step we have%
\begin{equation}
P_{j}^{2}-nQ_{j}^{2}=\left( -1\right) ^{j}C_{j},\,2\leq j\leq k  \label{ls2}
\end{equation}%
and in particular 
\begin{equation}
P_{j}^{2}\equiv \left( -1\right) ^{j}C_{j}\,\text{mod }n,\,2\leq j\leq k.
\label{ls3}
\end{equation}

Since from now on we shall be dealing exclusively with the congruence (\ref%
{ls3}) we see that it is sufficient in (\ref{ls1}) to calculate successive
values of $P_{j}$ mod $n$ giving an upper bound on the values of $P$. Then
in (\ref{ls3}) we can examine each even value of $j$ to find whether $C_{j}$
is an integer square. It might be thought that this extra checking would
slow down the process but in practice suitable squares are found relatively
quickly. Then if say $C_{j}=t^{2}$ for an even value of $j$ we can write (%
\ref{ls3}) as $P_{j}^{2}\equiv t\,^{2}$ mod $n,$ and proceed to find the
greatest common divisors of $n$ \ with $P_{j}\pm t.$ Of course failures of
Types I and II may still occur, but their number is very much reduced. We
give the following examples.

\textbf{Example 5}\label{e5} (i) Let $n=91,$ as in example \ref{e1} Then $%
P_{2}=10$ and $C_{2}=9=3^{2}$ giving $91=\left( 10-3\right) (10+3)=7.13$ as
required.

(ii) Let $n=1783647329$. Then after $2$ steps we find $C_{2}=1225$ $=35^{2}$%
and $P_{2}=168933.$ Hence%
\begin{eqnarray*}
168933^{2} &\equiv &35^{2}\text{ mod }n, \\
\left( 168933-35\right) \left( 168933+35\right) &\equiv &0\text{ mod }n \\
168898.168968 &\equiv &0\text{ mod }n
\end{eqnarray*}%
and the greatest common divisors of $n$ with the factors on the left are $%
84449$ and $21121$, giving the required factorisation\ of $n$ (into primes
as it happens). $\blacklozenge $

The example in (ii) above is given in \cite[p. 60]{Bressoud:1989:IEF} as one
which is particularly resistant to Fermat's method of factorisation by
writing $n$ as a difference of two squares.

The next table gives the results for this method which show, by comparison
with the table in the previous section, by how much failure rates have
fallen.

\begin{equation*}
\begin{tabular}{|l|l|l|l|l|}
\hline
Range of $\ n$ & Success & Type I fail & Type II fail & Time$(s)$ \\ \hline
$77\leq n\leq 1,000$ & $82\left( 89.1\%\right) $ & $2\left( 2.2\%\right) $ & 
$8\left( 8.7\%\right) $ & $0.03$ \\ \hline
$77\leq n\leq 2,000$ & $201\left( 91.0\%\right) $ & $4\left( 1.8\%\right) $
& $16\left( 7.2\%\right) $ & $0.08$ \\ \hline
$77\leq n\leq 5,000$ & $589\left( 91.0\%\right) $ & $13\left( 2.0\%\right) $
& $45\left( 7.0\%\right) $ & $0.23$ \\ \hline
$77\leq n\leq 10,000$ & $1297\left( 92.0\%\right) $ & $26\left( 1.8\%\right) 
$ & $87\left( 6.2\%\right) $ & $0.50$ \\ \hline
$77\leq n\leq 20,000$ & $2794\left( 92.2\%\right) $ & $68\left( 2.2\%\right) 
$ & $169\left( 5.6\%\right) $ & $1.11$ \\ \hline
\end{tabular}%
\end{equation*}

\bigskip

\textbf{Larger Values} In addition to the values of $n$ in this table we
have also considered batches of integers in the ranges $10^{m}<n<10^{m}+100$
for $10<m<20$. After removing multiples of 2,3,5 this gives around 20 values
to be tested in each batch, resulting in at most 1 failure in any batch,
which compares favourably with probabilistic algorithms over the same ranges.

Factorisation methods for Type I and II failures are in sections 4,5,and 6.

\section{Symmetry\label{sy}}

The sequence of partial quotients $\sqrt{n}=\left[ a_{0},a_{1},...,a_{k},...%
\right] $ (equivalently the sequence $\left( A_{j}\right) $ in the notation
above) with length $k$ is known to be symmetric in \ the sense that $%
a_{j}=a_{k-j}$ for $1\leq j\leq k-j.$ Thus when $k$ is even and $m=k/2$ we
have we have a central value $a_{m}$ with $a_{m-j}=a_{m+j}$ for $0\leq j<m.$
Similarly when $k$ is odd and $m=\left( k-1\right) /2$ there are two equal
central values $a_{m}=a_{m+1},$ with $a_{m-j}=a_{m+1+j}$ for $0\leq j<m.$

It is less well known that the sequences $\left( B_{j}\right) $ and $\left(
C_{j}\right) $ have similar symmetry properties which will give us useful
information about what happens at the the halfway point of the partial
fraction.

\begin{theorem}
$\label{sy1}$ Let $\left( B_{j}\right) $ and $\left( C_{j}\right) $ be
defined as in (\ref{ls1}), for the expansion with length $k$ for a given
input $n.$ Then for $1\leq j\leq k,$ 
\begin{eqnarray}
B_{j} &=&B_{k+1-j}\text{ and}  \label{sy2} \\
C_{j} &=&C_{k-j}.  \label{sy3}
\end{eqnarray}

In particular if $k$ is odd and $m=\left( k-1\right) /2$ then $C_{m}=C_{m+1}$
while if $k$ is even and $m=k/2$ then $B_{m}=B_{m+1}.$
\end{theorem}

This is proved in \cite[Section 3.4]{Jacobson:2009:JW} and will have a
number of important consequences, the first of which is as follows.

\begin{corollary}
\label{sy4}Suppose $k$ is odd and $m=\left( k-1\right) /2.$ Then $%
n=B_{m+1}^{2}+C_{m+1}^{2}.$
\end{corollary}

\begin{proof}
Immediate on substituting $C_{m}=C_{m+1}$\ into $B_{m+1}^{2}+C_{m}C_{m+1}=n$ 
$\ $(see the note following(\ref{ls1})).
\end{proof}

The following elegant result is crucial for what follows.

\begin{lemma}
\label{sy5}\textrm{(}M\"{a}rker\cite{Marker:1840:UP}\textrm{)} Let $k$ be
even and $m=k/2.$ If $C_{m}$ is even then either $C_{m}=2$ or C$_{m}/2$ is a
proper divisor of $n.$ If $C_{m}$ is odd then $C_{m}$ is a proper divisor of 
$n$
\end{lemma}

\begin{proof}
Notice that $C_{m}$ cannot equal 1 since this would indicate the end of the
expansion rather than the midpoint.

Since $k$ is even, we know from Theorem \ref{sy1} that $B_{m}=B_{m+1}$ and
hence from \ref{ls1}(iii) that $2B_{m}=A_{m+1}C_{m}.$ Since also $%
n=B_{j+1}^{2}+C_{j}C_{j+1}$ for all $j$ ,we deduce that with $j=m-1$ we have 
\begin{eqnarray}
n &=&B_{m}^{2}+C_{m-1}C_{m}  \notag \\
&=&\left( A_{m+1}C_{m}\right) ^{2}/4+C_{m-1}C_{m}  \label{sy6} \\
&=&\left( C_{m}/2\right) \left( A_{m+1}^{2}C_{m}/2+2C_{m-1}\right)  \notag \\
&=&\left( C_{m}/2\right) X\text{ }  \notag
\end{eqnarray}%
for some integer $X.$ It follows that if $C_{m}$ is even then $C_{m}/2$ is a
divisor of $n.$ Since clearly $X>2,$ it follows that $C_{m}/2$ is a proper
divisor in all cases except $C_{m}=2.$

If $C_{m}$ is odd then it follows from $2B_{m}=A_{m+1}C_{m}$ that $A_{m+1}$
is even and we can write $\left( \ref{sy6}\right) $ as $n=C_{m}\left( \left(
A_{m+1}/2\right) ^{2}C_{m}+C_{m-1}\right) =C_{m}Y$ for some integer $Y,$ and
since $Y>2,$ it follows that $C_{m}$ is a proper divisor of $n$ as required.
\end{proof}

This result enables us to sharpen our algorithm in the following way.

\textbf{Algorithm} For a positive integer $n$ which is not square, not prime
and not divisible 2,3 or 5, construct the sequences $\left( A_{j}\right)
,\left( B_{j}\right) ,\left( C_{j}\right) ,\left( P_{j}\text{ mod }n\right) $
as before.

For each value of $j\geq 2$ test $C_{j}$ as follows:

\qquad (i) is $C_{j}=C_{j-1}?$ \ If so then $n$ can be written as a sum of
squares, say $n=a^{2}+b^{2}$ as in Corollary \ref{sy4}, and we consider $n$
together with the corresponding $a,b$ in the next section \ref{gi}. Note
that all Type I failures from section \ref{ls} are dealt with here, leaving
Type II\ failures to be dealt with in (iii) below.

\qquad (ii) is $d:=\gcd \left( n,C_{j}\right) >1?$ If so the required
divisor $d$ is found and the algorithm terminates.

\qquad (iii) is $C_{j}=2?$ If so then we have a solution to the congruence $%
P_{j}^{2}\equiv 2(-1)^{j}$ mod $n,$ and we deal with $n$ together with the
corresponding values of $j,P_{j}$ in sections \ref{zmt} or \ref{zpt}
according to whether $j$ is odd or even respectively.

\qquad (iv) is $C_{j}=x^{2}$ for some (positive) integer $x$? If so then we
have a solution to the congruence $P_{j}^{2}\equiv (-1)^{j}x^{2}$ mod $n.$
If $j$ is odd we have $P_{j}^{2}+x^{2}\equiv 0$ mod $n$ and we deal with $n$
together with the corresponding $P_{j}$ and $x$ in section \ref{gi} as in
(i) above. Otherwise when $j$ is even we take the highest common factors of $%
n$ with $P\pm x$ as in section \ref{ls} to find a factorisation of $n.$ If
this is non-trivial then the algorithm terminates, otherwise we continue
with the next value of $j.$

\qquad

\textbf{Notes} 1. The condition $\gcd \left( n,C_{j}\right) >1$ is required
only to deal with the exceptional case which may occur in M\"{a}rker's Lemma
when $j=k/2.$ However it may it may occur earlier by chance and is quick to
test.

2. Only last part of (iv) requires us to move to the next value of $j$: all
other steps either send us to a different section, or terminate the
algorithm at once.

3. We have shown that when $k$ is odd then $C_{j}=C_{j+1}$ must occur at the
mid point $j=\left( k-1\right) /2$ , and that when $k$ is even that $C_{j}=2$
must occur when $j=k/2$ . Hence the iteration over $j$ must finish at the
midpoint if not sooner. This, together with the reduction of $P_{j}$ mod $n$
at each step, results in a substantial decrease in the number of values of $%
j $ to be considered, and the size of the values of P$_{j}$.

4. The above algorithm is stated for a single value of $n.$ If a range of
values is to be tested, then it seems more efficient to store the results of
(i), (iii) and (iv) in separate Lists which can then be referred to later
sections separately.

For instance when we consider the range 77$\leq n\leq 20,000$ we generate
three lists of which the first, (List 1), consists of 68 numbers between 533
and 19637 in which $n$ is either a sum of squares, together with the
associated $a,b$ with $n=a^{2}+b^{2}$ or a divisor of such a sum. These will
be dealt with in section \ref{gi}\ . Then there will be a second, (List 2a)
of 68 numbers between 731 and 19883 to be dealt with in section \ref{zmt}
when $C_{j}=2$,\thinspace\ $j$ is odd and $P_{j}^{2}\equiv -2$ mod n, and a
third, (List 2b) of 101 numbers between 119 and 19967 to be dealt with in
section \ref{zpt} when $C_{j}=2$,\thinspace\ $j$ is even, and $%
P_{j}^{2}\equiv 2$ mod n.

\section{Gaussian integers\label{gi}}

\textbf{Notation} Throughout this section the notation $\left( a,b\right) $
will denote the Gaussian integer $a+ib.$ All arithmetic operations together
with the Euclidean algorithm, and the formation of continued fractions - the
latter justified in the papers of Hurwitz, \cite{Hurwitz:1888:UEC},\cite%
{Hurwitz:1889:UAK} - will take place in $\mathbf{Z}\left( i\right) .$ We
shall use $\left( a,b\right) \cap (c,d)$ for the highest common factor of $%
\left( a,b\right) $ and $\left( c,d\right) $, uniquely defined since $%
\mathbf{Z}\left( i\right) $ is a Euclidean Domain.. When $x.y$ are real, ni$%
\left( x,y\right) $ will denote the nearest integer function in $\mathbf{Z}%
\left( i\right) $, equal to $\left( \left\{ x\right\} ,\left\{ y\right\}
\right) $ where $\left\{ x\right\} $ and $\left\{ y\right\} $ are the unique
integers in the intervals $[x-1/2,x+1/2).$and $[y-1/2,y+1/2)$

We shall consider the values of $n$ which arise in List1, that is from
stages (i) or (iv) in the algorithm in the previous section. Such numbers
are either equal to sums of squares or to divisors of such sums. For
instance if we consider $n=493$ by the method of section \ref{ls} we find
that $n$ is a divisor of $111^{2}+4=\left( 111,2\right) \left( 111,-2\right)
.$ To reduce this multiple of $493$ to a sum of squares we simply take $%
\left( 111,2\right) \cap \left( 493,0\right) =\left( 22,3\right) $ to obtain 
$493=22^{2}+$ $3^{2.}.$ When $n$ comes from part (i) we have seen from
Corollary \ref{sy4} that when $k$ is odd and $m=\left( k-1\right) /2$ then $%
n $ is given directly by $n=B_{m+1}^{2}+C_{m+1}^{2}.$

Hence in considering the entries in List 1, we may suppose that each value
of $n$ is known as a sum of squares, say $n=a^{2}+b^{2}=\left( a+ib\right)
\left( a-ib\right) $ for integers $a,b$ which we may assume positive without
loss of generality. To find a divisor of $n,$ we find divisors of $a+ib$ in $%
\mathbf{Z}\left( i\right) $ by calculating the continued fraction for $\sqrt{%
a+ib}$ and looking for square values of $C_{j}$ (\textit{not} the same as
the values of $C,$ etc. from the continued fraction for $\sqrt{n}).$ We may
use a numerical routine for the value of this square root, or find an exact
expression from the following:

\begin{lemma}
Let $\left( x+y\sqrt{r}\right) ^{2}=a+b\sqrt{r}$ with $a,b$ positive$.$ Then
for $r<0$ we may take 
\begin{equation*}
x=\sqrt{\frac{a+\sqrt{a^{2}-rb^{2}}}{2}\text{ }}\text{and }y=\frac{b}{2x}
\end{equation*}%
with both square roots positive.

If $r>0$ then we require also $a^{2}>rb^{2}$ and we may take

\begin{equation*}
x=\sqrt{\frac{a+\sqrt{a^{2}-rb^{2}}}{2}\text{ }}\text{and }y=\frac{b}{2x}%
\text{ similarly.}
\end{equation*}
\end{lemma}

\begin{proof}
Immediate on expanding $\left( x+y\sqrt{r}\right) ^{2}$ and equating
coefficients of $1,\sqrt{r}.$
\end{proof}

We shall need only the cases $r=-1,-2$ and $2$ in this result. With these
preliminaries, we now calculate the continued fraction for $\sqrt{\left(
a,b\right) }$ as in section \ref{ls} with the following modifications.

(i) Given $n=a^{2}+b^{2}$ and $\left( a,b\right) =a+ib$ to be factorised,
let $z=\sqrt{\left( a,b\right) }=\left( x,y\right) $ as in the Lemma$.$ Then
the (complex) continued fraction for $\sqrt{\left( a,b\right) }$ is given by
the sequences $\left( A_{j}\right) ,$ $\left( B_{j}\right) $ etc. which are
calculated as before, with $A_{1}=$ ni$\left( x,y\right) ,$ but for $j\geq
2, $ 
\begin{equation*}
A_{j+1}=\text{ ni}\left( \frac{z+B_{j}}{C_{j}}\right) \text{ }
\end{equation*}%
i.e. we do \emph{not} round to $A_{1}$ in the numerator here since the
customary simplification of replacing $\sqrt{n}$ by $\left\lfloor \sqrt{n}%
\right\rfloor $ in this formula does not work for complex integers.

(ii) $P_{j}$ should be reduced mod$\left( a,b\right) $ at each stage, where
for Gaussian integers 
\begin{equation*}
\left( c,d\right) \text{ mod }\left( a,b\right) :=\left( c,d\right) -\left(
a,b\right) \text{ni}\left( \left( c,d\right) /\left( a,b\right) \right) .
\end{equation*}

(iii) \ In $\mathbf{Z}\left( i\right) $ we \ have the property that $-1$ is
a square: $-1=i^{2}.$ Hence if $C_{j}$ is a square then so is $\left(
-1\right) ^{j}C_{j},$ irrespective of the parity of $j.$

(iv) Since $n=a^{2}+b^{2}=b^{2}+a^{2},$ we may start from $\left( a,b\right) 
$ or $\left( b,a\right) .$

We illustrate the method the following example.

\textbf{Example 6 }Let $n=533,$ the first entry on List 1. In the continued
fraction for $\sqrt{n}$ we find $B_{3}=2,$ C$_{2}=C_{3}=23$ and so $n=2^{2}+$
$23^{2.}.$ To factorise $\left( 2,23\right) $ we consider the continued
fraction for $\sqrt{\left( 2,23\right) }$ taking $\left( x,y\right)
=3.541...+i3.247...$ and so $A_{1}=\left( 4,3\right) .$ The first square
value of $C_{j}$ is $C_{6}=$ $\left( 1,0\right) $ and with $P_{6}$ mod $%
\left( 2,23\right) =\left( 1,0\right) $ \ we have only a trivial
factorisation. Thus we now try starting with $\left( 23,2\right) .$ We find $%
\left( x,y\right) =4.800...$ + $i\,0.208...$ and so $A_{1}=\left( 5,0\right)
.$ After 3 steps we find $P_{3}=\left( -1,24\right) $ , $\left( -1\right)
^{3}C_{3}=-\left( 0,-2\right) =\left( 0,2\right) =\left( 1,1\right) ^{2}.$
Then $\left( P_{3}+\left( 1,1\right) \right) \cap \left( 23,2\right) =\left(
1,0\right) $ and $\left( P_{3}-\left( 1,1\right) \right) \cap \left(
23,2\right) =\left( 23,2\right) $ and we have again trivial failure. But one
further step gives $P_{4}=\left( -105,-19\right) $ mod $\left( 23,2\right)
=\left( 10,-9\right) $ and $\left( -1\right) ^{4}C_{4}=\left( 3,4\right)
=\left( 2,1\right) ^{2}.$ Then $\left( P_{4}+\left( 2,1\right) \right) \cap
\left( 23,2\right) =\left( 3,-2\right) $ whose norm is $9+4=13,$ and $\left(
P_{4}-\left( 2,1\right) \right) \cap \left( 23,2\right) =\left( -5,-4\right) 
$ whose norm is $25+16=41,$ giving the required factorisation $n=533=13.41$. 
$\blacklozenge $

This example illustrates a method which works for 67 of the 68 entries in
List 1 in which one or other of $\left( a,b\right) $ or $\left( b,a\right) $
(sometimes both) factorise as above. The single exception is $n=3281$ for
which we find the following.

\textbf{Example 7 \ }Let $n=3281$ where the continued fraction for $\sqrt{%
3281}$ gives $B_{3}=16,$ $C_{3}=55$ and so $n=16^{2}+55^{2}$ or $%
55^{2}+16^{2}.$ However both $\left( 16,55\right) $ and $\left( 55,16\right) 
$ lead (by different routes) to $C_{10}=\left( 1,0\right) $ and $%
P_{10}\equiv \left( 1,0\right) $ and both lead to trivial factorisations so
the method fails. However before we abandon $n=3281$ entirely, a further
observation \ is worth making. It is (almost) evident by inspection that $%
n=3281=1681+1600=41^{2}+40^{2}$ so we have a different expression of $3281$
as a sum of squares. It would then be possible to attempt another
factorisation using $\sqrt{41+i40}$ but this is unnecessarily tedious.
Instead notice that $\left( 41,40\right) $ and $(16,55)$ are the products of
different choices of the complex prime factors of $3281.$Hence taking their
highest common factor we find $\left( 41,40\right) \cap \left( 16,55\right)
=\left( -1,4\right) $ whose norm is $1^{2}+4^{2}=17$ which is a proper
divisor of $3281$. The other divisor is $193$ which can be found either by
division or by taking for instance $\left( 41,40\right) \cap \left(
16,-55\right) =\left( 12,7\right) $ whose norm is $144+49=193.$ Thus the
decomposition $3281=17.193$ into primes is complete. $\blacklozenge $

The existence of further ways of writing a given composite integer as a sum
of squares is dealt with for instance in \cite[Ch. 19, p.164]%
{Mordell:1969:DT} and \cite[p, 101]{Landau:1927:VZ} where the latter gives
an explicit formula for the number of such representations. Methods for
finding numerical values for such representations do not seem to have been
studied. The search for different expressions of the value of $n$ under
consideration as $a^{2}\pm 2b^{2}$ is continued is the next two sections.

\section{$\mathbf{Z}\left( \protect\sqrt{-2}\right) $\label{zmt}}

\textbf{Notation} In this section the notation $\left( a,b\right) $ will
denote the element $a+b\sqrt{-2}$ of $\mathbf{Z}\left( \sqrt{-2}\right) .$
All arithmetic operations together with the Euclidean algorithm, and the
formation of continued fractions - the latter also justified in the papers
of Hurwitz, \cite{Hurwitz:1888:UEC},\cite{Hurwitz:1889:UAK} - will take
place in $\mathbf{Z}\left( \sqrt{-2}\right) .$ In particular $\mathbf{Z}%
\left( \sqrt{-2}\right) $satisfies Hurwitz' discreteness condition that only
a finite number of elements are in any bounded region of the complex plane.
We again use $\left( a,b\right) \cap (c,d)$ for the highest common factor of 
$\left( a,b\right) $ and $\left( c,d\right) $, again uniquely defined since $%
\mathbf{Z}\left( \sqrt{-2}\right) $ is a Euclidean Domain. When $x.y$ are
real, ni$\left( x,y\right) $ will denote the nearest integer function in $%
\mathbf{Z}\left( \sqrt{-2}\right) $, equal to $\left( \left\{ x\right\}
,\left\{ y\right\} \right) $ where $\left\{ x\right\} $ and $\left\{
y\right\} $ are the unique integers in the intervals $[x-1/2,x+1/2).$and $%
[y-1/2,y+1/2).$ Note that in contrast to $Z\left( i\right) ,$ $-1$ is not
here a square and that $\left( a,b\right) $ and $\left( b,a\right) $ are not
equivalent starting places.

Our input is List 2a of 68 numbers $n$ between 731 and 19883 when $C_{j}=2$%
,\thinspace\ $j$ is odd and $P_{j}^{2}\equiv -2$ mod $n$.

\textbf{Example 8 }(i) Let $n=731,$ where we have $n=729+2=27^{2}+2.1^{2}$
so we have to factorise $\left( 27,1\right) $ in $\mathbf{Z}\left( \sqrt{-2}%
\right) .$ Then $\sqrt{\left( 27,1\right) }=x+y\sqrt{-2}$ where 
\begin{equation*}
x=\sqrt{\frac{27+\sqrt{731}}{2}}=5.1979...,\text{ \thinspace }y=\frac{1}{2x}%
=0.09619...,
\end{equation*}%
and $A_{1}=\left( 5,0\right) .$ In the continued fraction for $\sqrt{\left(
27,1\right) }$ we find that on line 3 we have $\left( -1\right)
^{3}C_{3}=-\left( 1,2\right) =\left( 1,-1\right) ^{2}$ in $\mathbf{Z}\left( 
\sqrt{-2}\right) \,$and $P_{3}\equiv \left( 1,-26\right) $ mod $731$. Then $%
(P_{3}+\left( 1,-1\right) )\cap \left( 27,1\right) \equiv \left( 27,1\right) 
$ and the factorisation is trivial. We continue to find $C_{7}=C_{3}$ and $%
P_{7}\equiv \left( 5,-12\right) $ and so $(P_{7}+\left( 1,-1\right) )\cap
\left( 27,1\right) \equiv \left( 3,2\right) $ with $3^{2}+2.2^{2}=17$ which
divides $n.$ The other divisor is $43$ which is $731/17$ or by calculating $%
(P_{7}-\left( 1,-1\right) )\cap \left( 27,1\right) \equiv \left( 5,-3\right) 
$ whose norm is $25+2.9=43.$

(ii) Let $n=779,$ where $n$ is a proper divisor of $3433^{2}+2$. Then $%
\left( 3433,1\right) \cap \left( 779,0\right) =\left( 27,5\right) $ with $%
27^{2}+2.5^{2}=779$ and we have to factorise $\left( 27,5\right) $ in $%
\mathbf{Z}\left( \sqrt{-2}\right) .$ Then $\sqrt{\left( 27,5\right) }=x+y%
\sqrt{-2}$ where 
\begin{equation*}
x=\sqrt{\frac{27+\sqrt{779}}{2}}=5.2397...,\text{ \thinspace }y=\frac{1}{2x}%
=0.4721...,
\end{equation*}%
and $A_{1}=\left( 5,0\right) .$ In the continued fraction for $\sqrt{\left(
27,5\right) }$ we find that on line 7 we have $\left( -1\right)
^{7}C_{7}=-\left( 1,-2\right) =\left( 1,1\right) ^{2}$ in $\mathbf{Z}\left( 
\sqrt{-2}\right) \,$and $P_{7}\equiv \left( 12,0\right) $ mod $779$. Then $%
(P_{7}+\left( 1,1\right) )\cap \left( 27,5\right) \equiv \left( 1,3\right) $
with $1^{2}+2.3^{2}=19$ which is a divisor of $n,$ the other being $41.$ $%
\blacklozenge $

The method illustrated in this example succeeds for all elements of List 2a
with $3$ exceptions, namely $n=3827,7571,$ and 10307 which we will will
consider individually.

\textbf{Example 9 }(i) Let $n=3827$ which is a divisor of $3155^{2}+2$ Then
we have $\left( 3155,1\right) \cap \left( 3827,0\right) =\left( 57,17\right) 
$ to be factorised by considering $\sqrt{\left( 57,17\right) }.$

Proceeding as \ in the above examples we find $x=7.709...,\,y=1.102...$ and
so $A_{1}=\left( 8,1\right) .$ At $j=5$ we have $\left( -1\right)
^{j}C_{j}=\left( -1\right) ^{5}C_{5}=-\left( 7,4\right) =\left( 1,-2\right)
^{2}$ but this gives only a trivial factorisation and other squares fail
similarly. \ To find a different expression for $n$ as $a^{2}+2b^{2}$ we
could follow the example $7$ and look for values of $a,b$ which are nearly
equal. This quickly gives $a=33,\,b=37$ and to $\left( 57,17\right) \cap
\left( 33,37\right) =\left( 5,3\right) $ whose norm is $25+18=43$ which is a
divisor of $n$ and the factorisation follows.

(ii) Let $n=7571=87^{2}+2$ and we go directly to look for values of $a,b$
with $n=a^{2}+2b^{2}$ which are nearly equal. This takes a little longer to
find $a=39,\,b=55$ and $\left( 87,1\right) \cap \left( 39,55\right) =(9,4)$
whose norm is $81^{2}+2.4^{2}=113,$ giving $n=113.67.$

(iii) Let $n=10307$ which is a divisor of $\left( 4738\right) ^{2}+2$ and $%
\left( 4738,1\right) \cap \left( 10307,1\right) =\left( 87,37\right) $ with $%
87^{2}+2.37^{2}=10307.$ Here the required values of $a,b$ with $%
a^{2}+2b^{2}=10307$ are not at all equal and it is quicker to start from $%
a=1 $ and increase by (necessarily odd) values of $a$ to reach $a=15,\,b=71.$
Then $\left( 87,37\right) \cap \left( 15,71\right) =\left( 17,18\right) $
whose norm is $937$ giving $n=11.937.$ $\blacklozenge $

\section{$\mathbf{Z}\left( \protect\sqrt{2}\right) $\label{zpt}}

\textbf{Notation} In this section the notation $\left( a,b\right) $ will
denote the element $a+b\sqrt{2}$ of $\mathbf{Z}\left( \sqrt{2}\right) $ with
arithmetic operations etc. taking place in $\mathbf{Z}\left( \sqrt{2}\right)
.$ We \ should expect to encounter difficulties since the integers of $%
\mathbf{Z}\left( \sqrt{2}\right) $ are dense on the real line and hence do
not satisfy Hurwitz' discreteness condition. Given this, the success rate of
37\% for the square root method is surprisingly high. Failures are dealt
with using the method of searching for alternative solutions of $%
x^{2}-2y^{2}=n$ as in previous sections. For this section however there is a
systematic method which succeeds for all values of $n$ under consideration.

Our input is now the List 2b of 101 numbers $n$ between 119 and 19967 in
which $C_{j}=2$ with $j$ is even and $P_{j}^{2}\equiv 2$ mod $n$.

We begin with $2$ examples of the square root method.

\textbf{Example 10 }(i) Let $n=119,$ the first entry in List 2b. We have $%
n=11^{2}-2$ and so $\left( 11,1\right) $ is to be factorised in $\mathbf{Z}%
\left( \sqrt{2}\right) .$ For $\left( x+y\sqrt{2}\right) ^{2}=11$ +$\sqrt{2}$
we have 
\begin{equation*}
x=\sqrt{\frac{11+\sqrt{119}}{2}}=3.309...\text{, \ }y=\frac{1}{2x}=0.151...%
\text{and }A_{1}=\left( 3,0\right) .
\end{equation*}%
Then after only $2$ steps we find $C_{2}=\left( 1,0\right) ,\,P_{2}=\left(
19,-9\right) $ and $\left( P_{2}+1\right) \cap \left( 11,1\right) =\left(
1,1\right) ,$ a unit, while $\left( P_{2}-1\right) \cap \left( 11,1\right)
=\left( 11,1\right) $ so the method fails before it has got properly started.

(ii) To find an example where the method succeeds have to go to the seventh
element in List 2b, namely $n=1343.$ Here $623^{2}\equiv 2$ mod $n$ and $%
\left( 623,1\right) \cap \left( 1343,0\right) =\left( 41,13\right) $ with $%
\sqrt{\left( 41,13\right) }$ to be examined. Here $\left( 41,13\right) $ has
square root $x+y\sqrt{2}$ where 
\begin{equation*}
x=\sqrt{\frac{41+\sqrt{1343}}{2}}=6.230...\text{,\thinspace\ }y=\frac{13}{2x}%
=1.04...\text{and }A_{1}=\left( 6,1\right) .
\end{equation*}%
The expansion of $\sqrt{\left( 41,13\right) }$ gives $C_{6}=\left(
3,2\right) =\left( 1,1\right) ^{2}$ and $P_{6}=\left( -10,0\right) .$ Then $%
\left( P_{6}+\left( 1,1\right) \right) \cap \left( 41,13\right) =\left(
-9,1\right) $ whose norm is $79$ which divides $1343$ and the other factor
is $17$. $\blacklozenge $

Going through the whole of List 2b we find $\ 37$ successes and $64$
failures which, though more than might be expected, is still hardly
satisfactory. To get an idea of where possible additional integer solutions
of $x^{2}-2y^{2}=n$ could be found we return to Example 10(i) above in which 
$n=119.$ Notice that the solution $\left( 11,1\right) $ has the least value
of $y$ and hence is the nearest \ to the intercept $x=\sqrt{n}$ of the
hyperbola $x^{2}-2y^{2}=n$ with the $x-$axis. So we may consider possible
points on the corresponding hyperbola $x^{2}-2y^{2}=-n$ with small $x,$ or a
value of $y$ near to $y=\sqrt{n/2}.$ When $n=119$ we have $\sqrt{n/2}=7.713$
and this leads at once to $y=8$, $x=3$ with $x^{2}-2y^{2}=-n=-119.$ We
further observe that $\left( 11,1\right) \cap \left( 3,8\right) =\left(
5,2\right) $ whose norm is $25-8=17$ which is a divisor of $119,$ the other
being $7.$

We try an example from the opposite end of the list.

\textbf{Example 11} Let $n=19967.$ The $x-$ intercept is at $x=\sqrt{n}%
=141.304..$ so we start looking at the next smallest (necessarily odd) value
on $x^{2}-2y^{2}=n$, namely $x=143.$ Here $y$ is not an integer, but $x=145$
gives $y=23.$

For the $y-$ intercept at $\sqrt{n/2}=99.917...$ we start looking at $y=100.$
Both $100$ and $101$ give non-integer values of $x$ \ on $x^{2}-2y^{2}=-n,$
but $y=102$ gives $x=29.$

Finally $\left( 145,23\right) \cap \left( 29,102\right) =\left( 5,-16\right) 
$ whose norm is $487$ giving the factorisation $n=17.487.$ $\blacklozenge $

We find that this simple-minded method works for all $101$ numbers in List
2b (though surely not for all integers without restriction!) and we leave to
future investigators the question of whether in fact it works for all
integers of this type.

\textbf{Acknowledgements} It is a pleasure to thank Professor Hugh Williams
for friendly and informative correspondence, Dr Ingeborg Guba in
Stadtallendorf and the staff of the Bodleian Library for assistance with
finding copies of early German documents, and David Walker for indefatigable
programming.

\bigskip

7 Redgrove Park, Cheltenham, GL51 6QY, UK

\bigskip

peter.walker@cantab.net

\printbibliography

\end{document}
